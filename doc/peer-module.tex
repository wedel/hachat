%
% API Documentation for API Documentation
% Module peer
%
% Generated by epydoc 3.0.1
% [Sun Jan 27 17:55:27 2013]
%

%%%%%%%%%%%%%%%%%%%%%%%%%%%%%%%%%%%%%%%%%%%%%%%%%%%%%%%%%%%%%%%%%%%%%%%%%%%
%%                          Module Description                           %%
%%%%%%%%%%%%%%%%%%%%%%%%%%%%%%%%%%%%%%%%%%%%%%%%%%%%%%%%%%%%%%%%%%%%%%%%%%%

    \index{peer \textit{(module)}|(}
\section{Module peer}

    \label{peer}
central module which defines the behaviour of a Hachat Peer


%%%%%%%%%%%%%%%%%%%%%%%%%%%%%%%%%%%%%%%%%%%%%%%%%%%%%%%%%%%%%%%%%%%%%%%%%%%
%%                               Variables                               %%
%%%%%%%%%%%%%%%%%%%%%%%%%%%%%%%%%%%%%%%%%%%%%%%%%%%%%%%%%%%%%%%%%%%%%%%%%%%

  \subsection{Variables}

    \vspace{-1cm}
\hspace{\varindent}\begin{longtable}{|p{\varnamewidth}|p{\vardescrwidth}|l}
\cline{1-2}
\cline{1-2} \centering \textbf{Name} & \centering \textbf{Description}& \\
\cline{1-2}
\endhead\cline{1-2}\multicolumn{3}{r}{\small\textit{continued on next page}}\\\endfoot\cline{1-2}
\endlastfoot\raggedright \_\-\_\-p\-a\-c\-k\-a\-g\-e\-\_\-\_\- & \raggedright \textbf{Value:} 
{\tt None}&\\
\cline{1-2}
\end{longtable}


%%%%%%%%%%%%%%%%%%%%%%%%%%%%%%%%%%%%%%%%%%%%%%%%%%%%%%%%%%%%%%%%%%%%%%%%%%%
%%                           Class Description                           %%
%%%%%%%%%%%%%%%%%%%%%%%%%%%%%%%%%%%%%%%%%%%%%%%%%%%%%%%%%%%%%%%%%%%%%%%%%%%

    \index{peer \textit{(module)}!peer.Peer \textit{(class)}|(}
\subsection{Class Peer}

    \label{peer:Peer}
Peer Klasse


%%%%%%%%%%%%%%%%%%%%%%%%%%%%%%%%%%%%%%%%%%%%%%%%%%%%%%%%%%%%%%%%%%%%%%%%%%%
%%                                Methods                                %%
%%%%%%%%%%%%%%%%%%%%%%%%%%%%%%%%%%%%%%%%%%%%%%%%%%%%%%%%%%%%%%%%%%%%%%%%%%%

  \subsubsection{Methods}

    \label{peer:Peer:HistoryControl}
    \index{peer \textit{(module)}!peer.Peer \textit{(class)}!peer.Peer.HistoryControl \textit{(method)}}

    \vspace{0.5ex}

\hspace{.8\funcindent}\begin{boxedminipage}{\funcwidth}

    \raggedright \textbf{HistoryControl}(\textit{self}, \textit{neighbour}, \textit{historyList})

    \vspace{-1.5ex}

    \rule{\textwidth}{0.5\fboxrule}
\setlength{\parskip}{2ex}
    checks if the historyList from neighbour contains msgs wich are not in 
    own History. If so, these lostMsg Hahses will be pushed back and by 
    this the associated msgObjects are requested.

\setlength{\parskip}{1ex}
    \end{boxedminipage}

    \label{peer:Peer:__del__}
    \index{peer \textit{(module)}!peer.Peer \textit{(class)}!peer.Peer.\_\_del\_\_ \textit{(method)}}

    \vspace{0.5ex}

\hspace{.8\funcindent}\begin{boxedminipage}{\funcwidth}

    \raggedright \textbf{\_\_del\_\_}(\textit{self})

\setlength{\parskip}{2ex}
\setlength{\parskip}{1ex}
    \end{boxedminipage}

    \label{peer:Peer:__init__}
    \index{peer \textit{(module)}!peer.Peer \textit{(class)}!peer.Peer.\_\_init\_\_ \textit{(method)}}

    \vspace{0.5ex}

\hspace{.8\funcindent}\begin{boxedminipage}{\funcwidth}

    \raggedright \textbf{\_\_init\_\_}(\textit{self}, \textit{firstHost}={\tt None}, \textit{port}={\tt None}, \textit{name}={\tt \texttt{'}\texttt{temp}\texttt{'}}, \textit{ip}={\tt None}, \textit{testmode}={\tt False})

\setlength{\parskip}{2ex}
\setlength{\parskip}{1ex}
    \end{boxedminipage}

    \label{peer:Peer:addToHosts}
    \index{peer \textit{(module)}!peer.Peer \textit{(class)}!peer.Peer.addToHosts \textit{(method)}}

    \vspace{0.5ex}

\hspace{.8\funcindent}\begin{boxedminipage}{\funcwidth}

    \raggedright \textbf{addToHosts}(\textit{self}, \textit{addr})

    \vspace{-1.5ex}

    \rule{\textwidth}{0.5\fboxrule}
\setlength{\parskip}{2ex}
    check if already in hostlist otherwise add

\setlength{\parskip}{1ex}
    \end{boxedminipage}

    \label{peer:Peer:forwardMsg}
    \index{peer \textit{(module)}!peer.Peer \textit{(class)}!peer.Peer.forwardMsg \textit{(method)}}

    \vspace{0.5ex}

\hspace{.8\funcindent}\begin{boxedminipage}{\funcwidth}

    \raggedright \textbf{forwardMsg}(\textit{self}, \textit{msg}, \textit{Oneneigbour}={\tt None})

    \vspace{-1.5ex}

    \rule{\textwidth}{0.5\fboxrule}
\setlength{\parskip}{2ex}
    forwarding TextMessage, but not to initial sender if host is set, it 
    will only forward to this single host

\setlength{\parskip}{1ex}
    \end{boxedminipage}

    \label{peer:Peer:generateMsgParts}
    \index{peer \textit{(module)}!peer.Peer \textit{(class)}!peer.Peer.generateMsgParts \textit{(method)}}

    \vspace{0.5ex}

\hspace{.8\funcindent}\begin{boxedminipage}{\funcwidth}

    \raggedright \textbf{generateMsgParts}(\textit{self}, \textit{quant}={\tt 5}, \textit{length}={\tt 2000})

    \vspace{-1.5ex}

    \rule{\textwidth}{0.5\fboxrule}
\setlength{\parskip}{2ex}
    generates random TextMsgs, if length {\textgreater} 1000 there will be 
    more then one msg-part

\setlength{\parskip}{1ex}
    \end{boxedminipage}

    \label{peer:Peer:getHistory}
    \index{peer \textit{(module)}!peer.Peer \textit{(class)}!peer.Peer.getHistory \textit{(method)}}

    \vspace{0.5ex}

\hspace{.8\funcindent}\begin{boxedminipage}{\funcwidth}

    \raggedright \textbf{getHistory}(\textit{self}, \textit{neighbour}, \textit{initial}={\tt False})

    \vspace{-1.5ex}

    \rule{\textwidth}{0.5\fboxrule}
\setlength{\parskip}{2ex}
    request History from neigbour; initial is true for a initial history 
    exchange: this will skip pushing Hashes and immediately request 
    msg.objects

\setlength{\parskip}{1ex}
    \end{boxedminipage}

    \label{peer:Peer:maintenanceLoop}
    \index{peer \textit{(module)}!peer.Peer \textit{(class)}!peer.Peer.maintenanceLoop \textit{(method)}}

    \vspace{0.5ex}

\hspace{.8\funcindent}\begin{boxedminipage}{\funcwidth}

    \raggedright \textbf{maintenanceLoop}(\textit{self})

    \vspace{-1.5ex}

    \rule{\textwidth}{0.5\fboxrule}
\setlength{\parskip}{2ex}
    thread which runs regularly maintenance tasks

\setlength{\parskip}{1ex}
    \end{boxedminipage}

    \label{peer:Peer:processMessage}
    \index{peer \textit{(module)}!peer.Peer \textit{(class)}!peer.Peer.processMessage \textit{(method)}}

    \vspace{0.5ex}

\hspace{.8\funcindent}\begin{boxedminipage}{\funcwidth}

    \raggedright \textbf{processMessage}(\textit{self}, \textit{msg}, \textit{fromAddr})

    \vspace{-1.5ex}

    \rule{\textwidth}{0.5\fboxrule}
\setlength{\parskip}{2ex}
    processes the received messages

\setlength{\parskip}{1ex}
    \end{boxedminipage}

    \label{peer:Peer:pushHistory}
    \index{peer \textit{(module)}!peer.Peer \textit{(class)}!peer.Peer.pushHistory \textit{(method)}}

    \vspace{0.5ex}

\hspace{.8\funcindent}\begin{boxedminipage}{\funcwidth}

    \raggedright \textbf{pushHistory}(\textit{self}, \textit{neighbour}, \textit{quant})

    \vspace{-1.5ex}

    \rule{\textwidth}{0.5\fboxrule}
\setlength{\parskip}{2ex}
    push own List of History-Hashes to neighbour

\setlength{\parskip}{1ex}
    \end{boxedminipage}

    \label{peer:Peer:pushHosts}
    \index{peer \textit{(module)}!peer.Peer \textit{(class)}!peer.Peer.pushHosts \textit{(method)}}

    \vspace{0.5ex}

\hspace{.8\funcindent}\begin{boxedminipage}{\funcwidth}

    \raggedright \textbf{pushHosts}(\textit{self}, \textit{neighbour}, \textit{quant})

    \vspace{-1.5ex}

    \rule{\textwidth}{0.5\fboxrule}
\setlength{\parskip}{2ex}
    give Hosts from hostExchange to a neighbour

\setlength{\parskip}{1ex}
    \end{boxedminipage}

    \label{peer:Peer:pushMsgObjects}
    \index{peer \textit{(module)}!peer.Peer \textit{(class)}!peer.Peer.pushMsgObjects \textit{(method)}}

    \vspace{0.5ex}

\hspace{.8\funcindent}\begin{boxedminipage}{\funcwidth}

    \raggedright \textbf{pushMsgObjects}(\textit{self}, \textit{neighbour}, \textit{lostMsgHashes}={\tt None})

    \vspace{-1.5ex}

    \rule{\textwidth}{0.5\fboxrule}
\setlength{\parskip}{2ex}
    pushes requested msgObjects back to neighbour

\setlength{\parskip}{1ex}
    \end{boxedminipage}

    \label{peer:Peer:requestHosts}
    \index{peer \textit{(module)}!peer.Peer \textit{(class)}!peer.Peer.requestHosts \textit{(method)}}

    \vspace{0.5ex}

\hspace{.8\funcindent}\begin{boxedminipage}{\funcwidth}

    \raggedright \textbf{requestHosts}(\textit{self}, \textit{neighbour}, \textit{quant}={\tt None})

    \vspace{-1.5ex}

    \rule{\textwidth}{0.5\fboxrule}
\setlength{\parskip}{2ex}
    request Hosts from neighbour

\setlength{\parskip}{1ex}
    \end{boxedminipage}

    \label{peer:Peer:sendAll}
    \index{peer \textit{(module)}!peer.Peer \textit{(class)}!peer.Peer.sendAll \textit{(method)}}

    \vspace{0.5ex}

\hspace{.8\funcindent}\begin{boxedminipage}{\funcwidth}

    \raggedright \textbf{sendAll}(\textit{self}, \textit{msg})

    \vspace{-1.5ex}

    \rule{\textwidth}{0.5\fboxrule}
\setlength{\parskip}{2ex}
    send Message Object to all your Peers

\setlength{\parskip}{1ex}
    \end{boxedminipage}

    \label{peer:Peer:sendLoop}
    \index{peer \textit{(module)}!peer.Peer \textit{(class)}!peer.Peer.sendLoop \textit{(method)}}

    \vspace{0.5ex}

\hspace{.8\funcindent}\begin{boxedminipage}{\funcwidth}

    \raggedright \textbf{sendLoop}(\textit{self}, \textit{test}={\tt False})

    \vspace{-1.5ex}

    \rule{\textwidth}{0.5\fboxrule}
\setlength{\parskip}{2ex}
    send Message objects of all hosts from Queue as string

\setlength{\parskip}{1ex}
    \end{boxedminipage}

    \label{peer:Peer:sendText}
    \index{peer \textit{(module)}!peer.Peer \textit{(class)}!peer.Peer.sendText \textit{(method)}}

    \vspace{0.5ex}

\hspace{.8\funcindent}\begin{boxedminipage}{\funcwidth}

    \raggedright \textbf{sendText}(\textit{self}, \textit{text})

    \vspace{-1.5ex}

    \rule{\textwidth}{0.5\fboxrule}
\setlength{\parskip}{2ex}
    make TXTMessage out of text and send to all hosts

\setlength{\parskip}{1ex}
    \end{boxedminipage}

    \label{peer:Peer:startRecvLoop}
    \index{peer \textit{(module)}!peer.Peer \textit{(class)}!peer.Peer.startRecvLoop \textit{(method)}}

    \vspace{0.5ex}

\hspace{.8\funcindent}\begin{boxedminipage}{\funcwidth}

    \raggedright \textbf{startRecvLoop}(\textit{self})

    \vspace{-1.5ex}

    \rule{\textwidth}{0.5\fboxrule}
\setlength{\parskip}{2ex}
    general receive loop of a peer

\setlength{\parskip}{1ex}
    \end{boxedminipage}

    \index{peer \textit{(module)}!peer.Peer \textit{(class)}|)}
    \index{peer \textit{(module)}|)}
